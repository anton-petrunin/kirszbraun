\documentclass[oneside,a4paper]{article}
\usepackage{akp1}
\usepackage{hyper}
\usepackage{letters}
\usepackage{notation}
\usepackage{hyphenation}
\usepackage{envmnt}

\newcommand*\chapter[1]{}
\def\claim#1{%
\par%
\medskip%
\noindent%
\refstepcounter{thm}%
\hbox{\bf\boldmath \arabic{section}.\arabic{thm}. #1.}
\it\ %\ignorespaces
}
\def\endclaim{
\par
\medskip}

\begin{document}
\title{Alexandrov meets Kirszbraun}
\author{S. Alexander, V. Kapovitch, A. Petrunin}
\date{}
\maketitle

\begin{abstract}
We give a simplified proof of the generalized Kirszbraun theorem for Alexandrov spaces,
which is due to Lang and Schroeder.
We also discuss related questions, both solved and open. 
\end{abstract}

\section{Introduction}

Kirszbraun's theorem states that any \emph{short map} (i.e. 1-Lipschitz map) from a subset of Euclidean space to another in Euclidean space can be extended as a short map to the whole space.

This theorem was proved first by Kirszbraun in \cite{kirszbraun}.  Later it was reproved by Valentine in \cite{valentine-sphere} and \cite{valentine-kirszbraun}, where he also generalized it to pairs of Hilbert spaces of arbitrary dimension 
as well as pairs of spheres of the same dimension 
and pairs of hyperbolic spaces with the same curvature.
%S: I add a sentance:
J.~Isbel in \cite{isbell} studied target spaces that satisfy the above condition for any source space.

Valentine was also interested in pairs of metric spaces,
say $\spc{U}$ and $\spc{L}$,
which satisfy the above property, namely, given a subset $Q\subset\spc{U}$, 
any short map $Q\to\spc{L}$ 
can be extended to a short map $\spc{U}\to \spc{L}$.
It turns out that this property has a lot in common with the definition of Alexandrov spaces
(see theorems \ref{thm:kirsz+}, \ref{thm:kirsz-def} and \ref{thm:cba-kirsz-def}).
Surprisingly, this relationship was first discovered only in the 1990's;
it was first published by Lang and Schroeder in \cite{lang-schroeder}.
(The third author of this paper came to similar conclusions a couple of years earlier, and told it to the first author, but did not publish the result.)%???I am not sure we need to include this claim, feel free to remove it:) 

We slightly improve the results of Lang and Schroeder.
Our proof is based on the barycentric maps introduced by Kleiner in \cite{kleiner}.
The material of this paper will be included in the book on Alexandrov geometry that we are currently writing, but it seems useful to publish it now.


 \parbf{Structure of the paper.}
We introduce notations in Section~\ref{sec:prelim}.
In section \ref{sec:4pt} we give altternative definitions of Alexandrov spaces 
based on the Kirszbraun property for 4-point sets.
The generalized Kirszbraun theorem is proved in Section~\ref{sec:kirszbraun}.
In the sections \ref{sec:1+n} and \ref{sec:2n+2} we describe some comparison properties of finite subsets of Alexandrov spaces.
In Section~\ref{sec:kirszbraun:open} we discuss related open problems.
Appendices~\ref{sec:baricentric} and \ref{sec:helly} describe Kleiner's barycentric map and an analog of Helly's theorem for Alexandrov spaces.


\parbf{Historical remark. }%???Should we talk about that, A.???
 Not much is known about the author of this remarkable theorem.
The theorem appears in Kirszbraun's master's thesis which he defended in Warsaw University in 1930.
His name is Moj\.{z}esz and his second name is likely to be Dawid but is uncertain.
He was born either in 1903 or 1904 and died in a ghetto in 1942.
After university, 
he worked  as an actuary in an insurance company; 
\cite{kirszbraun} seems to be his only publication in mathematics. 

\parbf{Acknowledgment.} 
We want to thank S.~Ivanov, N.~Lebedeva and A.~Lytchak 
for useful comments
and pointing out misprints.
Also we want to thank L.~Grabowski for bringing to our attention the entry about Kirszbraun in the Polish Biographical Dictionary.

\section{Preliminaries}\label{sec:prelim}

In this section we mainly introduce our notations.

\parbf{Metric spaces.} Let $\spc{X}$ be a metric space.  The distance between two points $x,y\in\spc{X}$ will be denoted as $\dist{x}{y}{}$ or $\dist{x}{y}{\spc{X}}$.

Given $R\in[0,\infty]$ and $x\in \spc{X}$, the sets
\begin{align*}
\oBall(x,R)&=\{y\in \spc{X}\mid \dist{x}{y}{}<R\},
\\
\cBall[x,R]&=\{y\in \spc{X}\mid \dist{x}{y}{}\le R\}.
\end{align*}
are called respectively the \emph{open} and \emph{closed  ball} of radius $R$ with center at $x$.

A metric space $\spc{X}$ is called 
\emph{intrinsic}
if for any $\eps>0$ and any two points $x,y\in \spc{X}$ with $\dist{x}{y}{}<\infty$ there is an $\eps$-midpoint for $x$ and $y$;
i.e. there is a point $z\in \spc{X}$ such that $\dist{x}{z}{},\dist{z}{y}{}<\tfrac{1}{2}\cdot \dist[{{}}]{x}{y}{}+\eps$.


\parbf{Model space.}
$\Lob{m}{\kappa}$ denotes $m$-dimensional model space with curvature $\kappa$; 
i.e. the simply connected $m$-dimensional Riemannian manifold with constant sectional curvature $\kappa$.

Set $\varpi\kappa=\diam\Lob2\kappa$\index{$\varpi\kappa$}, so 
$\varpi\kappa=\infty$ if $\kappa\le0$ and $\varpi\kappa=\pi/\sqrt{\kappa}$ if $\kappa>0$.
(The letter $\varpi{}$ is a glyph variant of lower case $\pi$,
but is usually pronounced as \emph{pomega}.)

\parbf{Ghost of Euclid.} Let $\spc{X}$ be a metric space 
and $\II$ be a real interval.
A globally isometric map $\gamma\:\II\to \spc{X}$ will be called a \emph{unitspeed geodesic}. 
A unitspeed geodesic between $p$ and $q$ will be denoted by $\geod_{[p q]}$.
We consider $\geod_{[p q]}$ with parametrization starting at $p$; 
i.e. $\geod_{[p q]}(0)=p$ and $\geod_{[p q]}(\dist{p}{q}{})=q$.
The image of $\geod_{[p q]}$ will be denoted by $[p q]$ and called a \emph{geodesic}\index{geodesic}.

Also we will use the following short-cut notation:
\begin{align*}
\l] p q \r[&=[p q]\backslash\{p,q\},
&
\l] p q \r]&=[p q]\backslash\{p\},
&
\l[ p q \r[&=[p q]\backslash\{q\}.
\end{align*}

A metric space $\spc{X}$ is called 
\emph{geodesic}
if for any two points $x,y\in \spc{X}$ there is a geodesic $[x y]$ in $\spc{X}$.

Given a geodesic $[p q]$, we denote by $\dir{p}{q}$ its direction at $p$.
We may think of $\dir{p}{q}$ as belonging to the space of directions $\Sigma_p$ at $p$,
which in turn can be identified with the unit sphere in the tangent space $\T_p$ at $p$.
Further we set $\ddir{p}{q}=\dist[{{}}]{p}{q}{}\cdot\dir{p}{q}$;
it is a \emph{tangent vector} at $p$, that is, an element of $\T_p$.

For a triple of points $p,q,r\in \spc{X}$, a choice of triple of geodesics $([q r], [r p], [p q])$ will be called a \emph{triangle} and we will use the notation 
$\trig p q r=([q r], [r p], [p q])$.
If $p$ is distinct from $x$ and $y$, a  pair of geodesics $([p x],[p y])$ will be called a \emph{hinge}\index{hinge}, and  denoted by 
$\hinge p x y=([p x],[p y])$.

\parbf{Functions.}
A locally Lipschitz function $f$ on a metric space $\spc{X}$ is called $\lambda$-convex ($\lambda$-concave)
if for any geodesic $\geod_{[p q]}$ in $\spc{X}$ the real-to-real function 
$$t\mapsto f\circ\geod_{[p q]}(t)-\tfrac\lambda2\cdot t^2$$
is convex (respectively concave).
In this case we write $f''\ge \lambda$ (respectively $f''\le \lambda$).

A function $f$ is called \emph{strongly convex} (\emph{strongly concave})
if $f''\ge \delta$ (respectively $f''\le -\delta$) for some $\delta>0$.




\parbf{Model angles and triangles.} 
Let $\spc{X}$ be a metric space, 
$p,q,r\in \spc{X}$ 
and $\kappa\in\RR$. 
Let us define a \emph{model triangle} $\trig{\~p}{\~q}{\~r}$ 
(briefly, 
$\trig{\~p}{\~q}{\~r}=\modtrig\kappa(p q r)$) to be a triangle in the model plane $\Lob2\kappa$ such that
$$\dist{\~p}{\~q}{}=\dist{p}{q}{},
\ \ \dist{\~q}{\~r}{}=\dist{q}{r}{},
\ \ \dist{\~r}{\~p}{}=\dist{r}{p}{}.$$
If $\kappa\le 0$, the model triangle is said to be defined, since such a triangle always exists and is unique up to an isometry of $\Lob2\kappa$.
If $\kappa>0$, the model triangle is said to be defined if in addition
$$\dist{p}{q}{}+\dist{q}{r}{}+\dist{r}{p}{}< 2\cdot\varpi\kappa.$$
In this case the triangle also exists and is unique up to an isometry of $\Lob2\kappa$.

If for  $p,q,r\in \spc{X}$, the model triangle 
$\trig{\~p}{\~q}{\~r}=\modtrig\kappa(p q r)$ is defined 
and $\dist{p}{q}{},\dist{p}{r}{}>0$, then the  angle measure of 
$\trig{\~p}{\~q}{\~r}$ at $\~p$ will be called the \emph{model angle} of the triple $p$, $q$, $r$, and will be denoted by
$\angk\kappa p q r$.

\parbf{Curvature bounded below.}
We will denote by $\CBB{}{\kappa}$, complete intrinsic spaces  $\spc{L}$ with curvature $\ge\kappa$ in the sense of Alexandrov.
Specifically,  $\spc{L}\in \CBB{}{\kappa}$ if for any quadruple of points $p,x^1,x^2,x^3\in \spc{U}$ , we have
$$\angk\kappa p{x^1}{x^2}
+\angk\kappa p{x^2}{x^3}
+\angk\kappa p{x^3}{x^1}\le 2\cdot\pi.\eqlbl{Yup-kappa}$$
or at least one of the model angles $\angk\kappa p{x^i}{x^j}$ is not defined.

Condition \ref{Yup-kappa} will be called \emph{(1+3)-point comparison}.

According to Plaut's theorem \cite[Th. 27]{plaut:survey},
any space $\spc{L}\in \CBB{}{}$ is $G_\delta$-geodesic; 
that is, for any point $p\in \spc{L}$ there is a dense $G_\delta$-set $W_p\subset\spc{L}$ such that for any $q\in W_p$ there is a geodesic $[p q]$.

We will use two more equivalent definitions of $\CBB{}{}$ spaces (see \cite{AKP}).
Namely, a complete $G_\delta$-geodesic space is in $\CBB{}{}$ 
if and only if it satisfies either of following conditions: 
\begin{enumerate}

\item\label{POS-CBB-ref} (point-on-side comparison)
For any geodesic $[x y]$ and $z\in \l]x y\r[$, we have
$$\angk\kappa x p y\le\angk\kappa x p z; \eqlbl{POS-CBB}$$
or, equivalently, 
$$\dist{\~p}{\~z}{}\le \dist{p}{z}{},$$
where $\trig{\~p}{\~x}{\~y}=\modtrig\kappa(p x y)$, $\~z\in\l] \~x\~y\r[$, $\dist{\~x}{\~z}{}=\dist{x}{z}{}$.

\item (hinge comparison)
For any hinge $\hinge x p y$, the angle 
$\mangle\hinge x p y$ is defined and 
$$\mangle\hinge x p y\ge\angk\kappa x p y.$$
Moreover, if $z\in\l]x y\r[$, $z\not=p$ then for any two hinges $\hinge z p y$ and $\hinge z p x$ with common side $[z p]$
$$\mangle\hinge z p y + \mangle\hinge z p x\le\pi.$$
\end{enumerate}

We also use the following standard result in Alexandrov geometry, 
which follows from the discussion in the survey of Plaut \cite[8.2]{plaut:survey}.

\begin{thm}{Theorem}\label{thm:cbb-lin-part}
Let $\spc{L}\in \CBB{}{}$.
Given an array of points $(x^1,x^2\dots,x^n)$ in $\spc{L}$,
 there is a dense $G_\delta$-set $W\subset\spc{L}$ such that
for any $p\in W$, all the directions $\dir{p}{x^i}$ lie in 
an isometric copy of a unit sphere in $\Sigma_p$.
(Or, equivaletntly, all the vectors $\ddir{p}{x^i}$ lie in 
a subcone of the tangent space $\T_p$ which is isometric to Euclidean space.)
\end{thm}

\parbf{Curvature bounded above.}
We will denote by $\Cat{}{\kappa}$ 
the class of metric spaces $\spc{U}$ in which any two points at distance $<\varpi\kappa$ are joined by a geodesic, 
%(we say $\spc{U}$ is  \emph{geodesic at distance $<\varpi\kappa$})
and which have  curvature $\le\kappa$ in the following global sense of Alexandrov:  namely, for any quadruple of points $p^1,p^2,x^1,x^2\in \spc{U}$, we have
$$
 \angk{\kappa}{p^1}{x^1}{x^2} 
\le 
\angk{\kappa}{p^1}{p^2}{x^1}+\angk{\kappa}{p^1}{p^2}{x^2},
\ \t{or}\ 
\angk{\kappa} {p^2}{x^1}{x^2}\le \angk{\kappa} {p^2}{p^1}{x^1} + \angk{\kappa} {p^2}{p^1}{x^2},
\eqlbl{gokova:eq:2+2}$$
 or
one of the six model angles above 
is undefined.

The condition \ref{gokova:eq:2+2} will be called \emph{(2+2)-point comparison} (or \emph{(2+2)-point $\kappa$-comparison} 
if a confusion may arise).  

We denote the complete $\Cat{}{\kappa}$ spaces by $\cCat{}{\kappa}$. 

The following lemma is a direct consequence of the definition:

\begin{thm}{Lemma}\label{lem:cat-complete} 
Any complete intrinsic %S:length>intrinsic
 space $\spc{U}$  in which every quadruple $p^1,p^2,x^1,x^2$ satisfies 
the (2+2)-point $\kappa$-%S:it is defined now
comparison %\ref{gokova:eq:2+2} 
is   a $\cCat{}{\kappa}$ space (that is, any  two points at distance $<\varpi\kappa$ are  joined by a geodesic).  

In particular,  the completion of a 
$\Cat{}{\kappa}$ space again lies in $\Cat{}{\kappa}$.
\end{thm}

%\parbf{Remark.} For curvature bounded below, we find that $\CBB{}{\kappa}$ spaces are the appropriate focus.  For curvature bounded above, we consider $\Cat{}{\kappa}$ spaces; that is, we assume a geodesic-join property that follows from completeness rather than completeness itself, imposing the latter when needed. %The effectiveness of this approach is largely due to the following property of $\Cat{}{\kappa}$ spaces:
%S: This remark seems to be for me, but not for the reader, we will talk about this ``effectiveness'' again :) 

We have the following basic facts (see [1]):

\begin{thm}{Lemma}\label{lem:cat-unique} 
In a $\Cat{}{\kappa}$ space, geodesics of length $<\varpi\kappa$ are uniquely determined by, and continuously dependent on, their endpoint pairs.\end{thm}

\begin{thm}{Lemma}\label{lem:convex-balls}
In a $\Cat{}{\kappa}$ space, any open ball  $\oBall(x,R)$ of radius $R\le\varpi\kappa/2$  is convex, that is, $\oBall(x,R)$ contains every geodesic whose endpoints it contains.
\end{thm}

We also use an equivalent definition of $\Cat{}{\kappa}$ spaces (see \cite{AKP}).
Namely, a metric space  $\spc{U}$ in which any  two points at distance $<\varpi\kappa$ are  joined by a geodesic is a $\Cat{}{\kappa}$ space if and only if it satisfies the following condition: 
\begin{enumerate}

\item (point-on-side comparison)\label{cat-monoton}
for any geodesic $[x y]$ and $z\in \l]x y\r[$, we have
$$\angk\kappa x p y\ge\angk\kappa x p z,$$
or equivalently, 
$$\dist{\~p}{\~z}{}\ge \dist{p}{z}{}, \eqlbl{POS-CAT}$$
where $\trig{\~p}{\~x}{\~y}=\modtrig\kappa(p x y)$, $\~z\in\l] \~x\~y\r[$, $\dist{\~x}{\~z}{}=\dist{x}{z}{}$.


%\item (hinge comparison)\label{cat-hinge}
%for any hinge $\hinge x p y$, the angle 
%$\mangle\hinge x p y$ exists and
%$$\mangle\hinge x p y\le\angk\kappa x p y.$$

\end{enumerate}

We also use Reshetnyak's majorization theorem \cite{reshetnyak:major}.  
Suppose $\~\alpha$ is a simple closed curve of finite length  in $\Lob2{\kappa}$,
and $D\subset\Lob2{\kappa}$ is a closed region bounded by $\~\alpha$. If $\spc{X}$ is a metric space,  a length-nonincreasing map $F\:D\to\spc{X}$ is called \emph{majorizing} if it is length-preserving on $\~\alpha$.
In this case, we say that $D$ \emph{majorizes} the curve $\alpha=F\circ\~\alpha$ under the map $F$.


\begin{thm}{Reshetnyak's majorization theorem}
\label{thm:major}
Any closed curve $\alpha$ of length $<2\cdot \varpi\kappa$ in $\spc{U}\in\Cat{}{\kappa}$ is majorized by a convex region in $\Lob2\kappa$.
\end{thm}


\parbf{Ultralimit of metric spaces.}
Given a metric space $\spc{X}$, its ultrapower 
(i.e. ultralimit of constant sequence $\spc{X}_n=\spc{X}$) will be denoted as $\spc{X}^\o$;
here $\o$ denotes a fixed nonprinciple ultrafilter.
For definitions and properties of ultrapowers, 
we refer to a paper of Kleiner and Leeb \cite[2.4]{kleiner-leeb}.

We use the following facts about ultrapowers which easily follow from the definitions (see \cite{AKP} for details):
\begin{itemize}
\item $\spc{X}\in\cCat{}{\kappa}\  \Longleftrightarrow\  \spc{X}^\o\in\cCat{}{\kappa}$.
\item $\spc{X}\in\CBB{}{\kappa}\  \Longleftrightarrow\  \spc{X}^\o\in\CBB{}{\kappa}$.
\item $\spc{X}$ is intrinsic if and only if $\spc{X}^\o$ is geodesic.
\end{itemize}


Note that if $\spc{X}$ is \textit{proper} (namely, bounded closed sets are compact), then $\spc{X}$ and $\spc{X}^\o$ coincide.
Thus a reader interested only in proper spaces may ignore everything related to ultrapower in this article.

\input{kirszbraun.tex}
\appendix
\def\claim#1{%
\par%
\medskip%
\noindent%
\refstepcounter{thm}%
\hbox{\bf\boldmath \Alph{section}.\arabic{thm}. #1.}
\it\ %\ignorespaces
}
\section{Barycentric simplex}\label{sec:baricentric}

The barycentric simplex was introduced by Kleiner in \cite{kleiner};
it is a construction that works in a general metric space.
Roughly, it gives a $\kay$-\nospace dimensional submanifold for a given ``nondegenerate'' array of $\kay+1$ strongly convex functions.

Let us denote by $\Delta^\kay\subset \RR^{\kay+1}$\index{$\Delta^m$} 
the \emph{standard $\kay$-simplex}\index{standard simplex}; 
i.e. $\bm{x}=(x_0,x_1,\dots,x_n)\in\Delta^\kay$ if $\sum_{i=0}^\kay x_i=1$ and $x_i\ge0$ for all $i$.

Let $\spc{X}$ be a metric space 
and $\bm{f}=(f^0,f^1,\dots,f^\kay)\:\spc{X}\to \RR^{\kay+1}$ be a function array.
Consider the map $\spx{\bm{f}}\:\Delta^\kay\to \spc{X}$,\index{$\spx{\bm{f}}$} defined by 
$$\spx{\bm{f}}(\bm{x})=\argmin\sum_{i=0}^\kay x_i\cdot f^i,$$
where $\argmin f$\index{$\argmin$} denotes a point of minimum of $f$.
The map $\spx{\bm{f}}$ will be called a \emph{barycentric simplex}\index{barycentric simplex of function array} of $\bm{f}$.
In general, a barycentric simplex of a function array might be undefined and need not be unique. 

The name comes from the fact that if $\spc{X}$ is a Euclidean space 
and $f^i(x)\z=\tfrac{1}{2}\cdot\dist[2]{p^i}{x}{}$ for some array of points $\bm{p}=(p^0,p^1,\dots,p^\kay)$, 
then $\spx{\bm{f}}(\bm{x})$ is the barycenter of points $p^i$ with weights $x_i$.
 
A barycentric simplex $\spx{\bm{f}}$ 
for the function array $f^i(x)=\tfrac{1}{2}\cdot\dist[2]{p^i}{x}{}$
will also be called a \emph{barycentric simplex with vertices at} $\{p^i\}$\index{barycentric simplex with vertices at point array}.

It is clear from the  definition that if 
$\bm{\hat f}$ is a subarray of $\bm{f}$,
then $\spx{\bm{\hat f}}$ coincides with the restriction of $\spx{\bm{f}}$ to the corresponding face of $\Delta^\kay$.

The following theorem shows that the barycentric simplex is defined 
for an array of strongly convex functions on a complete geodesic space. 
In order to formulate the theorem, we need to introduce a partial order $\succcurlyeq$ on $\RR^{\kay+1}$.

\begin{thm}{Definition}\label{def:supset+succcurlyeq}
For two real arrays $\bm{v}$, $\bm{w}\in \RR^{\kay+1}$,
$\bm{v}=(v^0,v^1,\dots,v^\kay)$ 
and 
$\bm{w}=(w^0,w^1,\dots,w^\kay)$, 
we write
$\bm{v}\succcurlyeq\bm{w}$ if $v^i\ge w^i$ for each $i$.

Given a subset $Q\subset \RR^{\kay+1}$, define its \emph{superset}\index{superset}
\index{$\SupSet$}
$$\SupSet Q =\{\bm{v}\in\RR^\kay\mid\exists\, \bm{w}\in Q\ \t{such that}\ \bm{v}\succcurlyeq\bm{w}\}.$$

\end{thm}


\begin{thm}{Theorem on barycentric simplex}\label{thm:bary}
Assume $\spc{X}$ is a complete geodesic space and 
$\bm{f}\z=(f^0,f^1,\dots,f^\kay)\:\spc{X}\to\RR^\kay$ is an array of strongly convex and locally Lipschitz functions.

Then the barycentric simplex $\spx{\bm{f}}\:\Delta^\kay\to \spc{X}$
is uniquely defined and moreover:

\begin{subthm}{bary-Lip} $\spx{\bm{f}}$ is Lipschitz. 
\end{subthm}

\begin{subthm}{bary-iff} The set $\SupSet{\bm{f}(\spc{X})}\subset\RR^{\kay+1}$ is convex,
and
$p\in \spx{\bm{f}}(\Delta^\kay)$ if and only if
$\bm{f}(p)\in\Fr\l[\SupSet{\bm{f}(\spc{X})}\r]$.
In particular, $\bm{f}\circ\spx{\bm{f}}(\Delta^\kay)$ lies on a convex hypersurface in $\RR^{\kay+1}$.
\end{subthm}

\begin{subthm}{bary-embed} The restriction $\bm{f}|_{\spx{\bm{f}}(\Delta^\kay)}$  has  $C^{\frac{1}{2}}$-inverse.
\end{subthm}

\begin{subthm}{bary-R^n} 
The set $\mathfrak{S}=\spx{\bm{f}}(\Delta^\kay)\backslash\spx{\bm{f}}(\partial\Delta^\kay)$
is $C^{\frac{1}{2}}$-homeomorphic to an open domain in $\RR^\kay$.
\end{subthm}
\end{thm}

The set $\mathfrak{S}$ described above will be called \emph{Kleiner's spine}\index{Kleiner's spine} of $\bm{f}$.
If $\mathfrak{S}$ is nonempty, we say the barycentric simplex $\spx{\bm{f}}$ is \emph{nondegenerate}\index{nondegenerate}.

We precede the proof of the theorem with the following lemma.

\begin{thm}{Lemma}\label{lem:argmin(convex)}
Assume $\spc{X}$ is a complete geodesic metric space and let  $f\:\spc{X}\to\RR$ be a locally Lipschitz, strongly convex function.  Then the minimum point 
$p=\argmin f$ 
is uniquely defined.
\end{thm}

\parit{Proof.}
Assume that $x$ and $y$ are distinct minimum points of $f$. 
Then for the midpoint $z$ of a geodesic $[x y]$ we have
$$f(z)<f(x)=f(y),$$ 
a contradiction. 
It only remains to show existence.

Fix a point $p\in  \spc{X}$; 
let $\Lip\in\RR$ be a Lipschitz constant of $f$ in a neighborhood of $p$.
Without loss of generality, we can assume that $f$ is $1$-convex.
Consider function $\phi(t)=f\circ\geod_{[px]}(t)$.
Clearly $\phi$ is $1$-convex and $\phi^+(0)\ge -\Lip$.
Setting $\ell=\dist{p}{x}{}$, we get 
\begin{align*}
f(x)
&=
\phi(\ell)
\ge
\\
&\ge
f(p)-\Lip\cdot\ell+\tfrac{1}{2}\cdot\ell^2
\ge
\\
&\ge f(p)-\tfrac{1}{2}\cdot{\Lip^2}.
\end{align*}

In particular,
$$s
\df
\inf\set{f(x)}{x\in \spc{X}}
\ge
f(p)-\tfrac{1}{2}\cdot{\Lip^2}.$$
If $z$ is a midpoint of $[x y]$ then  
$$s\le f(z)
\le
\tfrac{1}{2}\cdot f(x)+\tfrac{1}{2}\cdot f(y)-\tfrac{1}{8}\cdot\dist[2]{x}{y}{}.
\eqlbl{mid-point}$$
Choose a sequence of points $p_n\in \spc{X}$  such that $f(p_n)\to s$.
Applying \ref{mid-point}, for $x\z=p_n$, $y\z=p_m$, we get that $(p_n)$ is a Cauchy sequence. 
Clearly, $p_n\to \argmin f$.
\qeds


 

\parit{Proof of theorem \ref{thm:bary}.}
Without loss of generality, we can assume that each $f^i$ is $1$-convex.
Thus, for any $\bm{x}\in\Delta^\kay$, 
the convex combination $\sum x_i\cdot f^i\:\spc{X}\to\RR$ is also $1$-convex.
Therefore, according to Lemma~\ref{lem:argmin(convex)}, $\spx{\bm{f}}(\bm{x})$ is defined.



\parit{(\ref{SHORT.bary-Lip}).} 
Since $\Delta^\kay$ is compact, it is sufficient to show that $\spx{\bm{f}}$ is locally Lipschitz.

For $\bm{x},\bm{y}\in\Delta^\kay$,
set 
\begin{align*}
f_{\bm{x}}
&=\sum x_i\cdot f^i,
&
f_{\bm{y}}
&=\sum y_i\cdot f^i,
\\
p
&=\spx{\bm{f}}(\bm{x}),
&
q
&=\spx{\bm{f}}(\bm{y}).
\end{align*}
Let $\ell=\dist[2]{p}{q}{}$.
Clearly 
$\phi(t)=f_{\bm{x}}\circ\geod_{[p q]}(t)$ takes its minimum at $0$ and
$\psi(t)=f_{\bm{y}}\circ\geod_{[p q]}(t)$ takes its minimum at $\ell$.
Thus $\phi^+(0)$, $\psi^-(\ell)\ge 0$%
\footnote{Here $\phi^\pm$ denotes ``signed one sided derivative''; i.e. 
$$\phi^\pm(t_0)=\lim_{t\to t_0\pm}\frac{\phi(t)-\phi(t_0)}{|t-t_0|}$$}%
.
From $1$-convexity of $f_{\bm{y}}$, we have
$\psi^+(0)+\psi^-(\ell)+\ell\le0$.

Let $\Lip$ be a Lipschitz constant for all $f^i$ in a neighborhood $\Omega\ni p$.
Then $\psi^+(0)\le \phi^+(0)+\Lip\cdot\|\bm{x}-\bm{y}\|_{{}_1}$, 
where $\|\bm{x}-\bm{y}\|_{{}_1}=\sum_{i=0}^\kay|x_i-y_i|$.
That is, given $\bm{x}\in\Delta^\kay$, there is a constant $\Lip$ such that
$$\dist{\spx{\bm{f}}(\bm{x})}{\spx{\bm{f}}(\bm{y})}{}
=
\ell\le \Lip\cdot\|\bm{x}-\bm{y}\|_{{}_1}$$
for any $\bm{y}\in\Delta^\kay$.
In particular, there is $\eps>0$ such that if $\|\bm{x}-\bm{y}\|_{{}_1},$ $\|\bm{x}-\bm{z}\|_{{}_1} <\eps$, then $\spx{\bm{f}}(\bm{y})$, $\spx{\bm{f}}(\bm{z})\in\Omega$. 
Thus, the same argument as above implies 
$$\dist{\spx{\bm{f}}(\bm{y})}{\spx{\bm{f}}(\bm{z})}{}
=
\ell\le \Lip\cdot\|\bm{y}-\bm{z}\|_{{}_1}$$
for any $\bm{y}$ and $\bm{z}$ sufficiently close to $\bm{x}$;
i.e. $\spx{\bm{f}}$ is locally Lipschitz.

\parit{(\ref{SHORT.bary-iff}).} The ``only if'' part is trivial, let us prove  the ``if''-part.

Note that convexity of $f^i$ implies that
for any two points $p,q\in \spc{X}$ and $t\in[0,1]$ we have
$$(1-t)\cdot\bm{f}(p)+t\cdot \bm{f}(q)
\succcurlyeq
\bm{f}\circ\geodpath_{[p q]}(t),
\eqlbl{n-convex}$$
where $\geodpath_{[p q]}$ is a geodesic path from $p$ to $q$; 
i.e. $\geodpath_{[p q]}(t)=\geod_{[p q]}(\tfrac{t}{\dist{p}{q}{}})$. 

From \ref{n-convex}, we have that $\SupSet[\bm{f}(\spc{X})]$ is a convex subset of $\RR^{\kay+1}$.
If 
$$\max_{i}\{f^i(q)\z-f^i(p)\}\ge0$$ 
for any $q\in \spc{X}$, then $\bm{f}(p)$ lies in the boundary of $\SupSet[\bm{f}(\spc{X})]$.
Take a supporting vector $\bm{x}\in\RR^{\kay+1}$ to $\SupSet[\bm{f}(\spc{X})]$ at $\bm{f}(p)$.
Thus $\bm{x}\not=\bm{0}$ and $\sum_i x_i\cdot[w^i-f^i(p)]\ge0$ for any 
$\bm{w}\in \SupSet[\bm{f}(\spc{X})]$. In particular, $\sum_i x_i\cdot v_i \ge 0$ for any $v=(v_1,\ldots, v_k)$ with all $v_i\ge 0$.  Hence $x_i\ge 0$ for all $i$ and 
$\bm{x}'=\frac{\bm{x}}{\|\bm{x}\|}_{{}_1}\in\Delta^\kay$. 
Thus $p=\spx{\bm{f}}(\bm{x}')$. 

\parit{(\ref{SHORT.bary-embed}).}
The restriction $\bm{f}|_{\spx{\bm{f}}(\Delta^\kay)}$ is Lipschitz.
Thus we only have to show that it has a  $C^{\frac{1}{2}}$-inverse.
Given $\bm{v}\in\RR^{\kay+1}$, consider the function 
$h_{\bm{v}}\: \spc{X}\to \RR$ given by
$$h_{\bm{v}}(p)=\max_{i}\{f^i(p)-v^i\}.$$
Define a map $\map \:\RR^{\kay+1}\to \spc{X}$ by
$\map (\bm{v})=\argmin h_{\bm{v}}$.

Clearly $h_{\bm{v}}$ is $1$-convex.
Thus, according to \ref{lem:argmin(convex)}, $\map (\bm{v})$ is uniquely defined for any $\bm{v}\in\RR^{\kay+1}$.
From (\ref{SHORT.bary-iff}), for any $p\in \spx{\bm{f}}(\Delta^\kay)$ we have
$\map \circ\bm{f}(p)=p$.

It remains to show that $\map $ is $C^{\frac{1}{2}}$-continuous.
Clearly, 
$$|h_{\bm{v}}-h_{\bm{w}}|
\le\|\bm{v}-\bm{w}\|_\subinfty
\df
\max_{i}\{|v^i-w^i|\},$$
for any $\bm{v},\bm{w}\in\RR^{\kay+1}$.
Set $p=\map (\bm{v})$ and $q=\map (\bm{w})$.
Since $h_{\bm{v}}$ and $h_{\bm{w}}$ are 1-convex,
\begin{align*}
h_{\bm{v}}(q)
&\ge 
h_{\bm{v}}(p)+\tfrac{1}{2}\cdot\dist[2]{p}{q}{},
&
h_{\bm{w}}(p)
&\ge 
h_{\bm{w}}(q)+\tfrac{1}{2}\cdot\dist[2]{p}{q}{}.
\end{align*}
Therefore,
$$\dist[2]{p}{q}{}\le 2\cdot\|\bm{v}-\bm{w}\|_\subinfty.$$
Hence the result.


\parit{(\ref{SHORT.bary-R^n}).} 
Let $S=\Fr\SupSet(\bm{f}(\spc{X}))$.
Note that orthogonal projection to the hyperplane $\WW^\kay$ in $\RR^{\kay+1}$ defined by equation $x_0+x_1+\dots+x_n=0$ gives a bi-Lipschits homeomorphism $S\to \WW^\kay$.

Clearly, $\bm{f}({\spx{\bm{f}}(\Delta^\kay)}\backslash\spx{\bm{f}}(\partial\Delta^\kay))$ 
is an open subset of $S$.
Hence the result.
\qeds
\section{Helly's theorem}\label{sec:helly}

\begin{thm}{Helly's theorem}\label{thm:helly}
Let $\spc{U}\in\cCat{}0$ 
and $\{K_\alpha\}_{\alpha\in \IndexSet}$ be an arbitrary collection of closed bounded convex subsets of $\spc{U}$.

If 
$$\bigcap_{\alpha\in \IndexSet}K_\alpha=\emptyset,$$
then there is an index array $\alpha_1,\alpha_2,\dots,\alpha_n\in \IndexSet$ such that
$$\bigcap_{i=1}^nK_{\alpha_i}=\emptyset.$$

\end{thm}

\parbf{Remarks.}
\begin{enumerate}[(i)]
\item In general, none of $K_\alpha$ might be compact. 
Thus the the statement is not completely trivial.
\item If $\spc{U}$ is a Hilbert space (not necessarily separable), 
then the above result is equivalent to the statement that a convex bounded set 
which is closed in the ordinary topology forms a compact set in the weak topology.

In fact, one can define the \emph{weak topology} on an arbitrary metric space, by taking exteriors of closed balls as its prebase.
Then the result above implies for $\spc{U}\in\cCat{}0$, any closed bounded convex set in $\spc{U}$ is compact in the  weak topology 
(this is very similar to the definition given by Monod in \cite{monod}).
\end{enumerate}

We present the proof of Lang and Shroeder from \cite{lang-schroeder}.

\begin{thm}{Lemma}\label{lem:closest point}
Let $\spc{U}\in\cCat{}0$.
Given a closed convex set $K\subset \spc{U}$ and a point $p\in \spc{U}\backslash K$, 
there is unique point $p^*\in K$ such that $\dist{p^*}{p}{}=\dist{K}{p}{}$. 
\end{thm}

\parit{Proof.}
Let us first prove uniqueness. 
Assume there are two points $y',y''\in K$ 
so that $\dist{y'}{p}{}=\dist{y''}{p}{}=\dist{K}{p}{}$.
Take $z$ to be the midpoint of $[y'y'']$. 
Since $K$ is convex, $z\in K$.
From comparison, we have that $\dist{z}{p}{}<\dist{y'}{p}{}=\dist{K}{p}{}$, a contradiction

The proof of existence is analogous.
Take a sequence  of points $y_n\in K$ 
such that $\dist{y_n}{p}{}\to \dist{K}{p}{}$.
It is enough to show that $(y_n)$ is a Cauchy sequence; 
thus one could take $p^*=\lim_n y_n$.

Assume $(y_n)$ is not Cauchy, then for some fixed $\eps>0$, 
we can choose two subsequences $(y_n')$ and $(y_n'')$ of $(y_n)$ 
such that 
$\dist{y'_n}{y''_n}{}\ge\eps$ for each $n$.
Set $z_n$ to be the midpoint of $[y'_ny''_n]$; from convexity we have $z_n\in K$.
From point-on-side comparison (see page \pageref{POS-CAT}), there is $\delta>0$ 
such that $\dist{p}{z_n}{}\le \max\{\dist{p}{y'_n}{},\dist{p}{y''_n}{}\}-\delta$. 
Thus 
$$\limsup_{n\to\infty}\dist{p}{z_n}{}<\dist{K}{x}{},$$ 
a contradiction\qeds

\parit{Proof of \ref{thm:helly}.} 
Assume the contrary. Then for any finite set $F\subset \IndexSet$,
$$K_{F}\df \bigcap_{\alpha\in F}K_{\alpha}\not=\emptyset.$$
We construct a point $z$ such that $z\in K_\alpha$ for each $\alpha\in \IndexSet$.
Thus we arrive at a contradiction since
$$\bigcap_{\alpha\in \IndexSet}K_\alpha=\emptyset.$$

Choose a point $p\in \spc{U}$ and set $r=\sup\dist{K_{F}}{p}{}$ where $F$ runs over all finite subsets of $\IndexSet$.
Let $p^*_F$  be the closest point on $K_{F}$ from $p$; 
according to Lemma \ref{lem:closest point}, $p^*_F$ 
exits and is unique.

Take a nested sequence of finite subsets 
$F_1\subset F_2\subset \dots$ of $\IndexSet$, such that $\dist{K_{F_n}}{p}{}\to r$.

Let us show that $p^*_{F_n}$ is a Cauchy sequence. 
Indeed, if not then for some fixed $\eps>0$, 
we can choose two subsequences $(y'_n)$ and $(y''_n)$ of $(p^*_{F_n})$ 
such that $\dist{y'_n}{y''_n}{}\ge\eps$.
Set $z_n$ to be midpoint of $[y'_ny''_n]$. 
From point-on-side comparison (see page \pageref{POS-CAT}), 
there is $\delta>0$ such that 
$\dist{p}{z_n}{}\le \max\{\dist{p}{y'_n}{},\dist{p}{y''_n}{}\}-\delta$.
Thus 
$$\limsup_{n\to\infty}\dist{p}{z_n}{}<r.$$
On the other hand, from convexity, each $F_n$ 
contains all $z_\kay$ with sufficiently large $\kay$, a contradiction.

Thus, $p^*_{F_n}$ converges and we can set $z=\lim_n p^*_{F_n}$.
Clearly $\dist{p}{z}{}=r$.

Repeat the above arguments for  the sequence $F_n'=F_n\cup \{\alpha\}$.
As a result, we get another point $z'$ such that $\dist{p}{z}{}=\dist{p}{z'}{}=r$ and 
$z,z'\in K_{F_n}$ for all $n$.
Thus, if $z\not=z'$ the midpoint $\hat z$ of $[zz']$ would belong to all 
$K_{F_n}$ and from comparison we would have $\dist{p}{\hat z}{}<r$, a contradiction.

Thus, $z'=z$; in particular 
$z\in K_\alpha$ for each $\alpha\in\IndexSet$.
\qeds
\begin{thebibliography}{52}
\bibitem{AKP}
Alexander, S., Kapovitch, V., Petrunin, A.,
\textit{Alexandrov geometry---foundations.}
Grad. Stud. Math., 236 (2024).

\bibitem{BBI} Burago, D.; Burago, Yu.; Ivanov, S., \textit{A course in metric geometry.} Grad. Stud. Math., 33. (2001).

\bibitem{gromov-CAT} Gromov, M., \textit{CAT($\kappa$)-spaces: Construction and Concentration,} Journal of Mathematical Sciences, Vol. 119, No. 2, (2004) 178--200.

\bibitem{isbell} Isbell, J. \textit{Six theorems about metric spaces}. Comment. Math. Helv. 39: 65--74.(1964).

\bibitem{kirszbraun} Kirszbraun, M. D., \textit{\"Uber die zusammenziehende und Lipschitzsche Transformationen,} Fundamenta Math. 22 (1934), 77--108.

\bibitem{kleiner} Kleiner, B.,  \textit{The local structure of length spaces
with curvature bounded above,} Math. Z. 231, 409--456 (1999).

\bibitem{kleiner-leeb}  Kleiner, B.; Leeb, B., \textit{Rigidity of quasi-isometries for symmetric spaces and Euclidean buildings.} Inst. Hautes \'Etudes Sci. Publ. Math. No. 86 (1997), 115--197 (1998).

\bibitem{lang-schroeder}  Lang, U.; Schroeder, V., \textit{Kirszbraun's theorem and metric spaces of bounded curvature,} Geom. funct. anal.
Vol. 7 (1997) 535--560.

\bibitem{LPZ} Lebedeva, N.;
Petrunin, A.; Zolotov, V., \textit{Bipolar comparison.}
Geom. Funct. Anal. 29 (2019), no. 1, 258--282.


\bibitem{monod} Monod, N. \textit{Superrigidity for irreducible lattices and geometric splitting,} J. of the AMS,
Vol. 19, Num. 4, October 2006, 781--814.

\bibitem{plaut:survey}
Plaut, C., \textit{Metric spaces of curvature $\ge k$.}
Handbook of geometric topology,  819--898, North-Holland, Amsterdam, 2002.

\bibitem{reshetnyak:major}
Reshetnyak, Yu. G.,
\textit{Nonexpanding maps in a space of curvature no greater than $K$},
Sibirskii Mat. Zh. 9 (1968), 918-928 (Russian).
English translation: \textit{Inextensible mappings in a space of curvature
no greater than $K$}, Siberian Math. Jour. 9 (1968),683-689.


\bibitem{sturm}
Sturm, K. T. \textit{Metric spaces of lower bounded curvature.}
Exposition. Math.  17  (1999),  no. 1, 35--47.

\bibitem{valentine-sphere}
Valentine F.A., Contractions in non-Euclidean spaces, Bull. Amer. Math. Soc. 50 (1944), 710--713.

\bibitem{valentine-kirszbraun}
Valentine, F. A., A Lipschitz condition preserving extension for a vector
function, Amer. J. Math. 67 (1945), 83--93.

\bibitem{vilms}
Vilms, J., \textit{Totally geodesic maps}.
J. Differential Geometry 4 1970 73--79.
\end{thebibliography}

\end{document}