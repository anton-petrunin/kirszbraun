\section{Barycentric simplex}\label{sec:baricentric}

The barycentric simplex was introduced by Kleiner in \cite{kleiner};
it is a construction that works in a general metric space.
Roughly, it gives a $\kay$-\nospace dimensional submanifold for a given ``nondegenerate'' array of $\kay+1$ strongly convex functions.

Let us denote by $\Delta^\kay\subset \RR^{\kay+1}$\index{$\Delta^m$} 
the \emph{standard $\kay$-simplex}\index{standard simplex}; 
i.e. $\bm{x}=(x_0,x_1,\dots,x_n)\in\Delta^\kay$ if $\sum_{i=0}^\kay x_i=1$ and $x_i\ge0$ for all $i$.

Let $\spc{X}$ be a metric space 
and $\bm{f}=(f^0,f^1,\dots,f^\kay)\:\spc{X}\to \RR^{\kay+1}$ be a function array.
Consider the map $\spx{\bm{f}}\:\Delta^\kay\to \spc{X}$,\index{$\spx{\bm{f}}$} defined by 
$$\spx{\bm{f}}(\bm{x})=\argmin\sum_{i=0}^\kay x_i\cdot f^i,$$
where $\argmin f$\index{$\argmin$} denotes a point of minimum of $f$.
The map $\spx{\bm{f}}$ will be called a \emph{barycentric simplex}\index{barycentric simplex of function array} of $\bm{f}$.
In general, a barycentric simplex of a function array might be undefined and need not be unique. 

The name comes from the fact that if $\spc{X}$ is a Euclidean space 
and $f^i(x)\z=\tfrac{1}{2}\cdot\dist[2]{p^i}{x}{}$ for some array of points $\bm{p}=(p^0,p^1,\dots,p^\kay)$, 
then $\spx{\bm{f}}(\bm{x})$ is the barycenter of points $p^i$ with weights $x_i$.
 
A barycentric simplex $\spx{\bm{f}}$ 
for the function array $f^i(x)=\tfrac{1}{2}\cdot\dist[2]{p^i}{x}{}$
will also be called a \emph{barycentric simplex with vertices at} $\{p^i\}$\index{barycentric simplex with vertices at point array}.

It is clear from the  definition that if 
$\bm{\hat f}$ is a subarray of $\bm{f}$,
then $\spx{\bm{\hat f}}$ coincides with the restriction of $\spx{\bm{f}}$ to the corresponding face of $\Delta^\kay$.

The following theorem shows that the barycentric simplex is defined 
for an array of strongly convex functions on a complete geodesic space. 
In order to formulate the theorem, we need to introduce a partial order $\succcurlyeq$ on $\RR^{\kay+1}$.

\begin{thm}{Definition}\label{def:supset+succcurlyeq}
For two real arrays $\bm{v}$, $\bm{w}\in \RR^{\kay+1}$,
$\bm{v}=(v^0,v^1,\dots,v^\kay)$ 
and 
$\bm{w}=(w^0,w^1,\dots,w^\kay)$, 
we write
$\bm{v}\succcurlyeq\bm{w}$ if $v^i\ge w^i$ for each $i$.

Given a subset $Q\subset \RR^{\kay+1}$, define its \emph{superset}\index{superset}
\index{$\SupSet$}
$$\SupSet Q =\{\bm{v}\in\RR^\kay\mid\exists\, \bm{w}\in Q\ \t{such that}\ \bm{v}\succcurlyeq\bm{w}\}.$$

\end{thm}


\begin{thm}{Theorem on barycentric simplex}\label{thm:bary}
Assume $\spc{X}$ is a complete geodesic space and 
$\bm{f}\z=(f^0,f^1,\dots,f^\kay)\:\spc{X}\to\RR^\kay$ is an array of strongly convex and locally Lipschitz functions.

Then the barycentric simplex $\spx{\bm{f}}\:\Delta^\kay\to \spc{X}$
is uniquely defined and moreover:

\begin{subthm}{bary-Lip} $\spx{\bm{f}}$ is Lipschitz. 
\end{subthm}

\begin{subthm}{bary-iff} The set $\SupSet{\bm{f}(\spc{X})}\subset\RR^{\kay+1}$ is convex,
and
$p\in \spx{\bm{f}}(\Delta^\kay)$ if and only if
$\bm{f}(p)\in\Fr\l[\SupSet{\bm{f}(\spc{X})}\r]$.
In particular, $\bm{f}\circ\spx{\bm{f}}(\Delta^\kay)$ lies on a convex hypersurface in $\RR^{\kay+1}$.
\end{subthm}

\begin{subthm}{bary-embed} The restriction $\bm{f}|_{\spx{\bm{f}}(\Delta^\kay)}$  has  $C^{\frac{1}{2}}$-inverse.
\end{subthm}

\begin{subthm}{bary-R^n} 
The set $\mathfrak{S}=\spx{\bm{f}}(\Delta^\kay)\backslash\spx{\bm{f}}(\partial\Delta^\kay)$
is $C^{\frac{1}{2}}$-homeomorphic to an open domain in $\RR^\kay$.
\end{subthm}
\end{thm}

The set $\mathfrak{S}$ described above will be called \emph{Kleiner's spine}\index{Kleiner's spine} of $\bm{f}$.
If $\mathfrak{S}$ is nonempty, we say the barycentric simplex $\spx{\bm{f}}$ is \emph{nondegenerate}\index{nondegenerate}.

We precede the proof of the theorem with the following lemma.

\begin{thm}{Lemma}\label{lem:argmin(convex)}
Assume $\spc{X}$ is a complete geodesic metric space and let  $f\:\spc{X}\to\RR$ be a locally Lipschitz, strongly convex function.  Then the minimum point 
$p=\argmin f$ 
is uniquely defined.
\end{thm}

\parit{Proof.}
Assume that $x$ and $y$ are distinct minimum points of $f$. 
Then for the midpoint $z$ of a geodesic $[x y]$ we have
$$f(z)<f(x)=f(y),$$ 
a contradiction. 
It only remains to show existence.

Fix a point $p\in  \spc{X}$; 
let $\Lip\in\RR$ be a Lipschitz constant of $f$ in a neighborhood of $p$.
Without loss of generality, we can assume that $f$ is $1$-convex.
Consider function $\phi(t)=f\circ\geod_{[px]}(t)$.
Clearly $\phi$ is $1$-convex and $\phi^+(0)\ge -\Lip$.
Setting $\ell=\dist{p}{x}{}$, we get 
\begin{align*}
f(x)
&=
\phi(\ell)
\ge
\\
&\ge
f(p)-\Lip\cdot\ell+\tfrac{1}{2}\cdot\ell^2
\ge
\\
&\ge f(p)-\tfrac{1}{2}\cdot{\Lip^2}.
\end{align*}

In particular,
$$s
\df
\inf\set{f(x)}{x\in \spc{X}}
\ge
f(p)-\tfrac{1}{2}\cdot{\Lip^2}.$$
If $z$ is a midpoint of $[x y]$ then  
$$s\le f(z)
\le
\tfrac{1}{2}\cdot f(x)+\tfrac{1}{2}\cdot f(y)-\tfrac{1}{8}\cdot\dist[2]{x}{y}{}.
\eqlbl{mid-point}$$
Choose a sequence of points $p_n\in \spc{X}$  such that $f(p_n)\to s$.
Applying \ref{mid-point}, for $x\z=p_n$, $y\z=p_m$, we get that $(p_n)$ is a Cauchy sequence. 
Clearly, $p_n\to \argmin f$.
\qeds


 

\parit{Proof of theorem \ref{thm:bary}.}
Without loss of generality, we can assume that each $f^i$ is $1$-convex.
Thus, for any $\bm{x}\in\Delta^\kay$, 
the convex combination $\sum x_i\cdot f^i\:\spc{X}\to\RR$ is also $1$-convex.
Therefore, according to Lemma~\ref{lem:argmin(convex)}, $\spx{\bm{f}}(\bm{x})$ is defined.



\parit{(\ref{SHORT.bary-Lip}).} 
Since $\Delta^\kay$ is compact, it is sufficient to show that $\spx{\bm{f}}$ is locally Lipschitz.

For $\bm{x},\bm{y}\in\Delta^\kay$,
set 
\begin{align*}
f_{\bm{x}}
&=\sum x_i\cdot f^i,
&
f_{\bm{y}}
&=\sum y_i\cdot f^i,
\\
p
&=\spx{\bm{f}}(\bm{x}),
&
q
&=\spx{\bm{f}}(\bm{y}).
\end{align*}
Let $\ell=\dist[2]{p}{q}{}$.
Clearly 
$\phi(t)=f_{\bm{x}}\circ\geod_{[p q]}(t)$ takes its minimum at $0$ and
$\psi(t)=f_{\bm{y}}\circ\geod_{[p q]}(t)$ takes its minimum at $\ell$.
Thus $\phi^+(0)$, $\psi^-(\ell)\ge 0$%
\footnote{Here $\phi^\pm$ denotes ``signed one sided derivative''; i.e. 
$$\phi^\pm(t_0)=\lim_{t\to t_0\pm}\frac{\phi(t)-\phi(t_0)}{|t-t_0|}$$}%
.
From $1$-convexity of $f_{\bm{y}}$, we have
$\psi^+(0)+\psi^-(\ell)+\ell\le0$.

Let $\Lip$ be a Lipschitz constant for all $f^i$ in a neighborhood $\Omega\ni p$.
Then $\psi^+(0)\le \phi^+(0)+\Lip\cdot\|\bm{x}-\bm{y}\|_{{}_1}$, 
where $\|\bm{x}-\bm{y}\|_{{}_1}=\sum_{i=0}^\kay|x_i-y_i|$.
That is, given $\bm{x}\in\Delta^\kay$, there is a constant $\Lip$ such that
$$\dist{\spx{\bm{f}}(\bm{x})}{\spx{\bm{f}}(\bm{y})}{}
=
\ell\le \Lip\cdot\|\bm{x}-\bm{y}\|_{{}_1}$$
for any $\bm{y}\in\Delta^\kay$.
In particular, there is $\eps>0$ such that if $\|\bm{x}-\bm{y}\|_{{}_1},$ $\|\bm{x}-\bm{z}\|_{{}_1} <\eps$, then $\spx{\bm{f}}(\bm{y})$, $\spx{\bm{f}}(\bm{z})\in\Omega$. 
Thus, the same argument as above implies 
$$\dist{\spx{\bm{f}}(\bm{y})}{\spx{\bm{f}}(\bm{z})}{}
=
\ell\le \Lip\cdot\|\bm{y}-\bm{z}\|_{{}_1}$$
for any $\bm{y}$ and $\bm{z}$ sufficiently close to $\bm{x}$;
i.e. $\spx{\bm{f}}$ is locally Lipschitz.

\parit{(\ref{SHORT.bary-iff}).} The ``only if'' part is trivial, let us prove  the ``if''-part.

Note that convexity of $f^i$ implies that
for any two points $p,q\in \spc{X}$ and $t\in[0,1]$ we have
$$(1-t)\cdot\bm{f}(p)+t\cdot \bm{f}(q)
\succcurlyeq
\bm{f}\circ\geodpath_{[p q]}(t),
\eqlbl{n-convex}$$
where $\geodpath_{[p q]}$ is a geodesic path from $p$ to $q$; 
i.e. $\geodpath_{[p q]}(t)=\geod_{[p q]}(\tfrac{t}{\dist{p}{q}{}})$. 

From \ref{n-convex}, we have that $\SupSet[\bm{f}(\spc{X})]$ is a convex subset of $\RR^{\kay+1}$.
If 
$$\max_{i}\{f^i(q)\z-f^i(p)\}\ge0$$ 
for any $q\in \spc{X}$, then $\bm{f}(p)$ lies in the boundary of $\SupSet[\bm{f}(\spc{X})]$.
Take a supporting vector $\bm{x}\in\RR^{\kay+1}$ to $\SupSet[\bm{f}(\spc{X})]$ at $\bm{f}(p)$.
Thus $\bm{x}\not=\bm{0}$ and $\sum_i x_i\cdot[w^i-f^i(p)]\ge0$ for any 
$\bm{w}\in \SupSet[\bm{f}(\spc{X})]$. In particular, $\sum_i x_i\cdot v_i \ge 0$ for any $v=(v_1,\ldots, v_k)$ with all $v_i\ge 0$.  Hence $x_i\ge 0$ for all $i$ and 
$\bm{x}'=\frac{\bm{x}}{\|\bm{x}\|}_{{}_1}\in\Delta^\kay$. 
Thus $p=\spx{\bm{f}}(\bm{x}')$. 

\parit{(\ref{SHORT.bary-embed}).}
The restriction $\bm{f}|_{\spx{\bm{f}}(\Delta^\kay)}$ is Lipschitz.
Thus we only have to show that it has a  $C^{\frac{1}{2}}$-inverse.
Given $\bm{v}\in\RR^{\kay+1}$, consider the function 
$h_{\bm{v}}\: \spc{X}\to \RR$ given by
$$h_{\bm{v}}(p)=\max_{i}\{f^i(p)-v^i\}.$$
Define a map $\map \:\RR^{\kay+1}\to \spc{X}$ by
$\map (\bm{v})=\argmin h_{\bm{v}}$.

Clearly $h_{\bm{v}}$ is $1$-convex.
Thus, according to \ref{lem:argmin(convex)}, $\map (\bm{v})$ is uniquely defined for any $\bm{v}\in\RR^{\kay+1}$.
From (\ref{SHORT.bary-iff}), for any $p\in \spx{\bm{f}}(\Delta^\kay)$ we have
$\map \circ\bm{f}(p)=p$.

It remains to show that $\map $ is $C^{\frac{1}{2}}$-continuous.
Clearly, 
$$|h_{\bm{v}}-h_{\bm{w}}|
\le\|\bm{v}-\bm{w}\|_\subinfty
\df
\max_{i}\{|v^i-w^i|\},$$
for any $\bm{v},\bm{w}\in\RR^{\kay+1}$.
Set $p=\map (\bm{v})$ and $q=\map (\bm{w})$.
Since $h_{\bm{v}}$ and $h_{\bm{w}}$ are 1-convex,
\begin{align*}
h_{\bm{v}}(q)
&\ge 
h_{\bm{v}}(p)+\tfrac{1}{2}\cdot\dist[2]{p}{q}{},
&
h_{\bm{w}}(p)
&\ge 
h_{\bm{w}}(q)+\tfrac{1}{2}\cdot\dist[2]{p}{q}{}.
\end{align*}
Therefore,
$$\dist[2]{p}{q}{}\le 2\cdot\|\bm{v}-\bm{w}\|_\subinfty.$$
Hence the result.


\parit{(\ref{SHORT.bary-R^n}).} 
Let $S=\Fr\SupSet(\bm{f}(\spc{X}))$.
Note that orthogonal projection to the hyperplane $\WW^\kay$ in $\RR^{\kay+1}$ defined by equation $x_0+x_1+\dots+x_n=0$ gives a bi-Lipschits homeomorphism $S\to \WW^\kay$.

Clearly, $\bm{f}({\spx{\bm{f}}(\Delta^\kay)}\backslash\spx{\bm{f}}(\partial\Delta^\kay))$ 
is an open subset of $S$.
Hence the result.
\qeds